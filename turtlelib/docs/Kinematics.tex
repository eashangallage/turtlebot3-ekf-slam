\documentclass{article}
\usepackage{amsmath}
\usepackage[a4paper, total={6in, 9in}]{geometry}

\begin{document}
\section*{Kinematics of a Differential Drive Robot}
\subsection*{Forward Kinematics}
We will begin by solving the forward kinematics problem.
Providing we know the pose of the robot at time $t$,
Given the wheel angles at a time $t+1$,
the forward kinematics allow us to find the pose of the
robot at $t+1$. Here we are assuming the robot follows a body twist for $t=1$
in arbitrary time units, so we are solving the forward kinematics
for 1 timestep.

In terms of transformations, with the world frame given as $T_W$ and
the body frame of the robot at time $t$ known to be $T_{WB}$,
we will use forward kinematics to find $T_{WB'}$ at time $t+1$
\\
\\
$D$ : half the distance between the wheel centers \\
\vspace{1mm}
$r$ : wheel radius \\
\vspace{1mm}
$\phi$ : wheel angle \\
\vspace{1mm}
$\dot V_{x_l}$ : velocity of the left wheel \\
\vspace{1mm}
$\dot V_{x_r}$ : velocity of the right wheel \\
\vspace{1mm}
$\dot \theta$ : $\theta$ component of the body twist \\
\vspace{1mm}
$\dot V_{x}$ : x component of the body twist \\
\vspace{1mm}
$\dot V_{y}$ : y component of the body twist \\

\noindent
Equations \ref{eq_left} and \ref{eq_right} descibe the relation
between the body twist and the wheel velocities for the left and
right wheels respectively


\begin{gather}
   \label{eq_left}
    \begin{bmatrix} \dot \theta \\ \dot V_{x_l} \\ \dot V_{y_l} \end{bmatrix}
        =
        \begin{bmatrix}
            1 & 0 & 0 \\
            -D & 1 & 0 \\
            0 & 0 & 1 \\
        \end{bmatrix}
        \begin{bmatrix} \dot \theta \\ \dot V_x \\ \dot V_y \end{bmatrix}
        =
    \begin{bmatrix} \dot \theta \\ -D \dot \theta + V_x \\ 0 \end{bmatrix}
\end{gather}

\begin{gather}
   \label{eq_right}
    \begin{bmatrix} \dot \theta \\ \dot V_{x_r} \\ \dot V_{y_r} \end{bmatrix}
        =
        \begin{bmatrix}
            1 & 0 & 0 \\
            D & 1 & 0 \\
            0 & 0 & 1 \\
        \end{bmatrix}
        \begin{bmatrix} \dot \theta \\ \dot V_x \\ \dot V_y \end{bmatrix}
        =
    \begin{bmatrix} \dot \theta \\ -D \dot \theta + V_x \\ 0 \end{bmatrix}
\end{gather}
\\
\\
\noindent
Since we are using traditional wheels that can only spin
forward or backward (no mecanum wheels or omniwheels) and we
are assuming that there is no slipping, $V_y = 0$. Furthermore,
the linear distance travelled by a wheel is given as $r \phi$. Since
here we consider that the robot follows the body twist for $t = 1$,
the wheel speed in this case is simply the difference in wheel angles,
$\Delta \phi_l = \dot \phi_l$ and $\Delta \phi_r = \dot \phi_r$.
Therefore, $V_{x_l} = r \dot \phi_l $, and $V_{x_r} = r \dot \phi_r$ measured
in radians per unit time.
Now solving Equations
\ref{eq_left} and \ref{eq_right} for $\dot \phi_l$ and $\dot \phi_r$
we get:

\begin{equation} \label{eq_fk1}
   \begin{split}
      \dot \phi_l & = \frac{-D \dot \theta + V_x}{r} \\
      \dot \phi_r & = \frac{D \dot \theta + V_x}{r} \\
   \end{split}
\end{equation}

\noindent
Now we will want to find the body twist that brings the robot
from its starting configuration to the new configuration, or in
other words we want the transformation $T_{BB'}$. For this we need
the body twist in which we can find from Equations \ref{eq_fk1}.
Solving for $\dot \theta$, then for $V_x$, and finally recalling
that since we assert there be no slipping, $V_y = 0$, we obtain
the following equations for the three components of the body twist.
\begin{equation}
   \begin{split}
      \dot \theta & = \frac{r}{2D} \left(\dot \phi_r - \dot \phi_l\right) \\[10pt]
      \mathcal{V}_x & = \frac{r}{2} \left(\dot \phi_r + \dot \phi_l\right) \\[10pt]
      \mathcal{V}_y & = 0
   \end{split}
\end{equation}

\noindent
Finally, we can integrate this twist using the \emph{rigid2D} library
to obtain $T_{BB'}$, and from there we obtain $T_{WB'} = T_{WB} \cdot T_{BB'}$.
The pose of the robot can now be easily obtained from $T_{WB'}$.

\subsection*{Inverse Kinematics}
Given some desired body twist, solving the inverse kinematics problem
allows us to determine the wheel velocities needed to achieve that twist.
Assumptions regarding slipping and only considering a unit time step as in
the forward kinematics problem hold here as well.

\begin{gather}
   \label{eq_ik}
    \begin{bmatrix} \dot \phi_l \\ \dot \phi_r \end{bmatrix}
        =
        \frac{1}{r}
        \begin{bmatrix}
            -D & 1 & 0 \\
            D & 1 & 1 \\
        \end{bmatrix}
         \begin{bmatrix} \dot \theta \\ \dot V_x \\ \dot V_y \end{bmatrix}
        \\[10pt]
   \label{eq_ik2}
    \begin{bmatrix} \dot \phi_l \\ \dot \phi_r \end{bmatrix}
        =
        \frac{1}{r}
        \begin{bmatrix}
            -D \dot \theta + V_x  \\
            D \dot \theta + V_x + V_y \\
        \end{bmatrix}
\end{gather}

\noindent
Equation \ref{eq_ik2} shows the wheel velocities, $\dot \phi_l$ and $\dot \phi_r$
needed to achieve the desired body twist. Note that as we have stated
previously, for no slipping to occur, $V_y = 0$

\end{document}

